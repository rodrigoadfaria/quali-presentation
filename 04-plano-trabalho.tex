\section{Plano de Trabalho}

\begin{frame}{Disciplinas cursadas}
\begin{table}[!htpb]
\centering
\begin{small}
\setlength{\tabcolsep}{1.5pt}

\begin{tabular}{|l|c|}\hline
 \thb{Disciplina} & \thb{Término} \\ \hline
 Aprendizagem Computacional: Modelos, Algoritmos e Aplicações & jun/2015 \\ \hline
 Introdução à Computação Gráfica	& jun/2015 \\ \hline
 Visão e Processamento de Imagens - Parte I & jun/2015 \\ \hline
 Análise de Algoritmos & nov/2015 \\ \hline
 Métodos de Aprendizagem em Visão Computacional & nov/2015 \\ \hline
 Linguagens, Autômatos e Computabilidade & jun/2016 \\ \hline
 Programação Orientada a Objetos & jun/2016 \\\hline

\end{tabular}
\end{small}
\end{table}    
\end{frame}

%------------------------------------------------------
\begin{frame}{Atividades previstas}
\begin{enumerate}
    \item Revisar leituras adicionais
    \item Incorporar novos conjuntos de dados
    \item Investigar características passíveis de uso em classificadores de dados intervalares
    \item Desenvolver ferramentas para dar suporte aos experimentos subsequentes
    \item Elaborar novos experimentos com base nas ferramentas desenvolvidas e conjuntos de dados estabelecidos
    \item Analisar os resultados e reportá-los no projeto de pesquisa
    \item Publicar resultados em artigos científicos
    \item Escrever a dissertação
\end{enumerate}
\end{frame}

%------------------------------------------------------
\begin{frame}{Cronograma}
\begin{table}[!htpb]
\centering
% definindo o tamanho da fonte para small
\begin{small} 
  
% redefinindo o espaçamento das colunas
\setlength{\tabcolsep}{5pt} 

% \cline semelhante ao \hline, indicando as colunas com a linha horizontal
% \multicolumn{12}{c|}{Meses} indica que doze colunas serão mescladas
\begin{tabular}{|c|c|c|c|c|c|c|c|c|c|c|c|c|}\hline
 & \multicolumn{11}{c|}{Meses 2016/2017}\\ \cline{2-12}
\raisebox{1.5ex}{Atividade} & dez & jan & fev & mar & abr & mai & jun & jul & ago & set & out \\ \hline

1 & x & x & x &   &   &   &   &   &   &   &   \\ \hline
2 & x & x &   &   &   &   &   &   &   &   &   \\ \hline
3 &   & x & x &   &   &   &   &   &   &   &   \\ \hline
4 &   &   & x & x & x & x &   &   &   &   &   \\ \hline
5 &   &   &   &   & x & x & x &   &   &   &   \\ \hline
6 &   &   &   &   &   & x & x & x &   &   &   \\ \hline
7 &   &   &   &   &   &   & x & x &   &   &   \\ \hline
8 &   &   &   &   &   &   &   & x & x & x & x \\ \hline

\end{tabular} 
\end{small}
\end{table}
\end{frame}